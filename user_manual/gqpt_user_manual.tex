\documentclass{article}
\usepackage[utf8]{inputenc}
\usepackage{amsmath}
\usepackage{array}
\usepackage{biblatex}
\usepackage{multicol,caption}
\usepackage{graphicx}
\usepackage{breqn}
\usepackage[margin=1.0in]{geometry}

\usepackage{listings}
\lstset{frame=none,
  language=Python,
  aboveskip=3mm,
  belowskip=3mm,
  showstringspaces=false,
  columns=flexible,
  basicstyle={\small\ttfamily},
  numbers=none,
  numberstyle=\tiny\color{gray},
  keywordstyle=\color{blue},
  commentstyle=\color{dkgreen},
  stringstyle=\color{mauve},
  breaklines=true,
  breakatwhitespace=true,
  tabsize=3
}

\newenvironment{Figure}
  {\par\medskip\noindent\minipage{\linewidth}}
  {\endminipage\par\medskip}

\title{User guide for Python upgrade of \\Gemini Queue Planning Tool prototype} 
\author{Matt Bonnyman \\ \textit{Gemini Observatory}, La Serena, Chile \\ \textit{University of Victoria}, British Columbia, Canada \\\\
Bryan Miller \\ \textit{Gemini Observatory}, La Serena, Chile}
\date{\today}

\begin{document}

\maketitle

\vspace{5mm}

\subsubsection*{Software Summary}
This software is a continuation of the Gemini Queue Planning Prototype Tool developed by Bryan Miller in 2004.  From May-August 2018, the original IDL software was converted to Python and various new features were added.

\subsubsection*{Document Purpose}
This document demonstrates the use of the Python Gemini Queue Planning Tool.  It is intended as a guide for using the program's various features, as well as a resource for understanding the software data structures and input file formats.  This document contains examples of how the software can be used to examine scheduling algorithms; examine observation weighting schemes; perform queue scheduling; simulate incoming targets of opportunity; simulate changing viewing conditions. 

\subsubsection*{Intended Audience}
This document is intended for use by Gemini's operations and software staff.   

\clearpage
\tableofcontents

\clearpage
% ------------------------------------------------------------------------------------------------------------------------------------------------------------------------------
\section{Introduction}
% -------------------------------------------------------------------------------------
\subsection{Context}
\label{sec:context}
In 2005, Bryan Miller developed a \textit{Gemini Queue Planning Tool Prototype} (GQPT).   This software uses real or simulated weather conditions to generate nightly observing schedules consisting of the highest priority observable programs.  For several years, Gemini's queue coordinators (QCs) used this software to prepare nightly observing plans.  

\subsection{Installation}


\subsection{Contribution}
\label{sec:contribution}
The original GQPT software has been converted from IDL to Python and several new features have been added.  The new features aim to more accurately simulate observing at the Gemini telescopes.  In addition, a small program for examining observation weighting functions was developed.  This program is called the \textit{Weight Function Plotting Tool} (WFPT).  The upgrade process consisted of the following tasks...\\
\begin{enumerate}
\item Convert IDL prototype to Python.
\item Revamp work flow and data structures to accommodate new features.
\item Implement methods for evaluation of scheduling results.
\item Implement methods for evaluation of scheduling algorithms.
\item Develop a method for evaluating observation weighting schemes.
\item Implement a schedule of available instruments and instrument components.
\item Incorporate Gemini observation time constraints.
\item Develop a target of opportunity simulation.
\item Develop a changing sky condition simulation. 
\end{enumerate}

\subsection{Software capabilities}
\label{sec:uses}
The following is a list of the ways that the upgraded GQPT software and WFPT can be used - note that some of these features may be used simultaneously and have several customizable parameters.  Most parameters can be defined using the command line or from within the program menu.\\
\begin{enumerate}
\item Generate observing plans over several nights.
\item Simulate targets of opportunity.
\item Simulate changing viewing conditions.
\item Examine steps made to assemble nightly plans.
\item Examine observation weighting functions.
\end{enumerate}

\subsection{Gemini Queue Planning Tool Prototype (gqpt.py)}
{\centering
 \includegraphics[width=0.9 \columnwidth]{wf2}\\
 \captionof{figure}{New software work flow}\label{fig:pythonworkflow}
}
\vspace{4mm}

\subsection{Weight Function Plotting Tool (wfpt.py)}
The weight function plotting tool uses a simplified version of the GQPT workflow.  In particular, it only needs to construct the necessary data for a single night of scheduling and does not include any simulation features (Section \ref{sec:wfpt}).

\section{GQPT}
\label{sec:gqpt}
This section will provide examples of the various ways that the program can be used.  An observation file, program file, and instrument schedule file are always required to use the software.  These files are summarized in Section \ref{sec:fileformats}.  

\subsection{Help guide}

\begin{lstlisting}
>>> python gqpt.py -h

                                Gemini Adaptive Queue Planning Tool
*****************************************************************************************************               

    REQUIRED
    --------
    otfile                  OT catalog filename.

    prfile                  Program status filename.

    instcal                 Instrument calendar filename.

    OPTIONAL
    --------
    -s   --startdate        Start date 'YYYY-MM-DD' [DEFAULT=current].

    -e   --enddate          End date 'YYYY-MM-DD' [DEFAULT=startdate]. End date must be before
                            start date.  If no end date is provided, the scheduling period will
                            default to a single night.

    -dst --daylightsavings Toggle daylight savings time [DEFAULT=False].

    -dt  --gridsize         Size of time-grid spacing [DEFAULT=0.1hr].

    -o   --observatory      Observatory site [DEFAULT='gemini_south']. Accepts the following:
                            1. 'gemini_north' (or 'MK' for Mauna Kea)
                            2. 'gemini_south' (or 'CP' for Cerro Pachon)

    -l   --logfile          Logfile name [DEFAULT='gaqptDDMMYY-hh:mm:ss.log'].

    -t  --toofile           Target of opportunity observation models filename [DEFAULT=None].

    -tp  --tooprob          Probability of incoming ToOs during the night [DEFAULT=0].

    -tm  --toomax           Maximum number of potential ToOs during the night [DEFAULT=4].

    -cp  --condprob         Probability of a sky conditions changing during the night [DEFAULT=0].

    -cm  --condmax          Maximum number of potential sky conditions changes during the night 										 [DEFAULT=4].  

    -p   --plantype         Scheduling algorithm type [DEFAULT='Priority']. 

                            Conditions (if distribution=False):
    -iq  --iq               Image quality constraint [DEFAULT=70%].
    -cc  --cc               Cloud cover constraint   [DEFAULT=50%].
    -wv  --wv               Water vapor constraint   [DEFAULT=Any].

    -d   --distribution     Random viewing conditions from distribution [DEFAULT=False]. Accepts the 										 following:
                            1. 'random' (or 'r').  Generate conditions from uniform distribution.
                            2. 'variant' (or 'v').  Randomly select one of several variants.

                            Wind conditions:
    -dir --direction        Wind direction [DEFAULT=330deg].
    -vel --velocity         Wind velocity [DEFAULT=5m/s].

    -rw  --randwind         Random wind conditions [DEFAULT=False]. 
                            Means and standard deviations at sites:
                                Cerro Pachon : dir=(330 +/- 20)deg, vel=(5 +/- 3)m/s
                                Mauna Kea    : dir=(330 +/- 20)deg, vel=(5 +/- 3)m/s                            

    -pp  --planplots        Show airmass plot of nightly plan [DEFAULT=False].

    -ip  --iterplots        Show airmass plot after each iteration of the plan (when simulating 
                            incoming ToO and changing sky conditions) [DEFAULT=False].

    -bp  --buildupplots     Show airmass plot after each time an observation is added 
                            to the plan [DEFAULT=False].

    -sp  --skyplots         Show sky conditions plot [DEFAULT=False]. 

    -wp  --windplots        Show wind condition plot [DEFAULT=False]. 

    -u   --update           Download up-to-date IERS(International Earth Rotation and Reference 										  Systems) data.

    -rs  --seed             Random seed number for random number generation [DEFAULT=1000].

*****************************************************************************************************                        

positional arguments:
  otfile
  prfile
  instcal

optional arguments:
  -h, --help            show this help message and exit
  -s STARTDATE, --startdate STARTDATE
  -e ENDDATE, --enddate ENDDATE
  -dst, --daylightsavings
  -dt GRIDSIZE, --gridsize GRIDSIZE
  -l LOGFILE, --logfile LOGFILE
  -o OBSERVATORY, --observatory OBSERVATORY
  -t TOOFILE, --toofile TOOFILE
  -tp TOOPROB, --tooprob TOOPROB
  -tm TOOMAX, --toomax TOOMAX
  -cp CONDPROB, --condprob CONDPROB
  -cm CONDMAX, --condmax CONDMAX
  -p PLANTYPE, --plantype PLANTYPE
  -iq IQ, --iq IQ
  -cc CC, --cc CC
  -wv WV, --wv WV
  -d DISTRIBUTION, --distribution DISTRIBUTION
  -dir DIRECTION, --direction DIRECTION
  -vel VELOCITY, --velocity VELOCITY
  -rw, --randwind
  -pp, --planplots
  -ip, --iterplots
  -bp, --buildupplots
  -sp, --skyplots
  -wp, --windplots
  -u, --update
  -rs SEED, --seed SEED

\end{lstlisting}

  
\subsection{Launching from command line}
\label{sec:launch}
Most of the simulation parameters can be defined in either the command line or the simulation menu.  However, the observatory location, range of dates, daylight savings time, three required input files, and the time grid spacing cannot be changed once the program is launched.\\

The most basic way to launch the program is to provide only the three required input files.  For example...
\begin{lstlisting}
>>> python gqpt.py observations.txt exechours.txt  instschedule.txt 
\end{lstlisting}

The software uses a default one-night observing period on the current date at the Gemini South location and with a time-grid spacing of 0.1hr.\\

The user may prepare a one-night observing period for any date by providing only a start date.
\begin{lstlisting}
>>> python gqpt.py observations.txt exechours.txt  instschedule.txt --start 2018-01-01
\end{lstlisting}

An end date can be applied to prepare a scheduling period of any length.  For example, the following would prepare a six-month scheduling period at Gemini North.
\begin{lstlisting}
>>> python gqpt.py observations.txt exechours.txt  instschedule.txt --start 2018-01-01 --end 2018-06-01 --observatory MK
\end{lstlisting}

Once a scheduling period is prepared, simulations can be run any number of times from the program menu without re-launching the software.  The menu will appear and prompt the user to continue once all time-grid data, Sun data, Moon data, observation information, instrument information, program information, and target data is prepared.

\subsection{Scheduling}
\label{sec:scheduling}

The user can define the remaining simulation parameters from the program menu.  Here is an example of a simulation prepared using the default settings...
\begin{lstlisting}
>>> python gqpt.py observations.txt exechours.txt  instschedule.txt 
...observatory site, time_zone, utc_to_local
...scheduling period dates
...time data and grids
...Sun data
...Moon data
...observations
...programs
...instrument calendar
...target calendar
...timing windows (convert time constraints)
...timing windows (twilights)
...timing windows (organize into nights)
...timing windows (instrument calendar)
...time window indices
...target data
...timing windows (elevation constraint)
 
	---------------------------------------------------------------------
	                  Gemini Adaptive Queue Planning Tool
	---------------------------------------------------------------------
	
	Dates: 2018-07-13 to 2018-07-13
	Number of nights: 1
	Daylight savings time: False
	
	Observatory: 
		Site:          gemini_south
		Height:        2750.0000 m
		Longitude:     -70.7367 deg
		Latitude:      -30.2407 deg

	Options:
	--------
	1.        Log file                                     		gqpt13Jul13-19:57:56.log               
	2.        ToO file                                     		None                                    
	3.        ToO probability                              		0.0                                     
	4.        Max. number of ToOs per night                		4                                       
	5.        Conditions (iq,cc,wv)                        		(70%, 50%, Any)                         
	6.        Conditions from distribution type            		None                                    
	7.        Wind conditions (dir, vel)                   		(330.0deg, 5.0m/s)                      
	8.        Generate random wind conditions              		False                                   
	9.        Probability of condition change              		0.0                                     
	10.       Max. number of condition changes per night   		4                                       
	11.       Show plan plots                              		False                                    
	12.       Show airmass plot of each plan iteration     		False                                   
	13.       Show airmass plots of plan building up       		False                                   
	14.       Show sky conditions plots                    		False                                   
	15.       Show wind condition plots                    		False                                   

	dir       Show files in current directory                                                      

 Press enter to run or select an option: 
\end{lstlisting}

If run using the default settings, a nightly plan will be produced.

\begin{lstlisting}
										-- Generating plan for night of 2018-07-13 --
	
	Sky conditions (iq, cc, wv):       (70%, 50%, Any)
	Wind conditions (dir., vel.):      330.0 deg, 5.0 m / s
	
	Solar midnight (UTC):              2018-07-14 04:48
	Solar midnight (local):            2018-07-14 00:48
	
	Sun ra:                            113.2 deg
	Sun dec:                           21.72 deg
	
	Moon ra:                           129.04 deg
	Moon dec:                          18.87 deg
	Moon fraction:                     0.02
	Moon phase:                        2.87 rad
	
	
	                                   -- 2018-07-13 schedule --
	
	Obs. ID       Target      RA     Dec.   Instr  UTC    LST    Start  End    Dur.   AM    HA     Completed 
	-------       ------      --     ---    -----  ---    ---    -----  ---    ---    --    --     --------- 
	12 deg.twi.                                    22:54  13.65  18:54                                       
	-FT-108-25    T2_ucd1     202.9  -41.9  GMOS-  23:06  13.85  19:06  19:30  0.4 h  1.02  0.32   True      
	-FT-103-31    hip 67620   208.1  -24.4  GPI    23:42  14.45  19:42  20:30  0.8 h  1.01  0.58   True      
	-FT-108-47    T7_ucd3     204.0  -43.0  GMOS-  00:30  15.25  20:30  20:54  0.4 h  1.09  1.65   True      
	-FT-108-37    T2_ucd2     203.4  -41.2  GMOS-  00:54  15.66  20:54  21:18  0.4 h  1.13  2.10   True      
	-FT-108-31    T2_ucd3     202.8  -40.9  GMOS-  01:18  16.06  21:18  21:48  0.5 h  1.18  2.53   True      
	-FT-108-43    T1_ucd1     201.9  -43.6  GMOS-  01:48  16.56  21:48  22:18  0.5 h  1.28  3.09   True      
	-FT-108-19    T6_ucd1     202.7  -45.5  GMOS-  02:18  17.06  22:18  22:42  0.4 h  1.38  3.54   True      
	8A-LP-11-7    133P        275.5  -21.4  GMOS-  03:24  18.16  23:24  23:48  0.4 h  1.01  -0.21  True      
	A-Q-103-23    HR Del      310.8  19.23  GMOS-  03:48  18.56  23:48  00:18  0.5 h  1.91  -2.16  True      
	8A-Q-106-9    J2217+0029  334.6  0.58   GMOS-  04:18  19.07  00:18  00:54  0.6 h  1.76  -3.25  True      
	A-Q-106-23    J2159+0005  330.1  0.18   GMOS-  04:54  19.67  00:54  02:06  1.2 h  1.42  -2.34  True      
	A-Q-106-21    J2141-0016  325.6  -0.19  GMOS-  06:06  20.87  02:06  03:18  1.2 h  1.18  -0.84  True      
	A-Q-106-15    J2249+0047  342.5  0.9 d  GMOS-  07:18  22.07  03:18  04:30  1.2 h  1.19  -0.77  True      
	A-Q-106-37    J2343+0038  356.0  0.75   GMOS-  08:48  23.58  04:48  05:24  0.6 h  1.17  -0.16  True      
	A-Q-106-25    J2209-0055  332.5  -0.84  GMOS-  09:30  0.28   05:30  06:06  0.6 h  1.35  2.11   True      
	12 deg. twi.                                   10:36  1.38   06:36                                       
\end{lstlisting}

The program will then return to the simulation menu.  Each simulation will have the results saved to a time-stamped log file.  These files contain the simulation parameters, initial and final program statuses, as well as the nightly plans and events (i.e. incoming targets of opportunity, viewing condition changes).\\

\subsubsection{Plan plots}

Without re-launching the program, the same one-night simulation can be re-run with plot outputs.  Figure \ref{fig:1night} shows an airmass and altitude-azimuth plot of the 2018-07-13 plan.  This feature will generate plots for each nightly plan in the scheduling period.  The plots will pop-up after each plan is generated and prompt the user to resume scheduling.

{\centering
 \includegraphics[width= \columnwidth]{am_aa_1night}\\
 \captionof{figure}{Airmass(left) and altitude-azimuth(right) of the 2018-07-13 plan}\label{fig:1night}
}
\vspace{4mm}

\subsubsection{Plan build-up plots}
The plan build-up feature is intended for use on very short scheduling periods.  It is a tool for examining the program's scheduling algorithms.  Figure \ref{fig:planbuild} shows the 2018-07-13 plan building up using the $priority$ scheduling algorithm.  This algorithm selects the observation with the highest weight and schedules it at the most optimal time in the night(i.e. maximum of the weighting function integrated over the observation period).

{\centering
 \includegraphics[width= \columnwidth]{planbuild}\\
 \captionof{figure}{First four steps of 2018-07-13 plan assembly}\label{fig:planbuild}
}
\vspace{4mm}

\subsubsection{Plan iteration plots}
This feature is used when incoming targets of opportunity or changing viewing conditions are being simulated.  When these events occur throughout the night, the plan may be interrupted and regenerated.  These plots will display each iteration of the plan as the simulation progress through the night (Section \ref{sec:examplesim}).

\subsubsection{Condition plots}
The sky conditions plot feature will show decimal percentiles throughout the night for image quality, cloud condition, and water vapour (Figure \ref{fig:skywind}).  The sky background percentile is specific to each target and therefore is not displayed with this feature.  Sky background conditions for individual targets can be viewed using the \textit{Weight Function Plotting Tool} (Figure \ref{fig:skyplot2}).  Wind condition plots display the direction and velocity.  Currently there is no simulation of changing wind conditions so these conditions remain constant throughout the night.

\subsection{Simulations}
\label{sec:sims}
Currently the GQPT has methods for simulating changing viewing conditions and incoming targets of opportunity (ToOs).  Each of these simulators uses an event a probability and maximum event number.  For example, if the ToO probability and event maximum are set to 0.2 and 4, up to 4 ToOs will arrive to the queue at random times throughout the night.  

\subsubsection{Viewing conditions simulation}
The current conditions simulation uses random condition changes.  This is by no means a realistic simulation method, but it has put the framework in place for more realistic simulations to be implemented.  When a condition change event occurs, the simulation will determine whether or not the current observation can be continued.  If so, the plan will remain unchanged.  In the case that the conditions worsen, the plan may be interrupted and the remainder of the night rescheduled for the new conditions.

\subsubsection{Target of opportunity simulation}
The target of opportunity simulator requires a file of ToO observation types.  This file is formatted exactly the same as the observation catalog file.  However, the timing constraints for ToOs must have a value of -1 as the constraint start time.  The software will set this start time as the time of arrival to the queue.  Secondly, each ToO must have a program name but no observation name.  The observation name must be set to 'null'.  As ToO events are generated, each observation will be assigned a number.  For example, if a ToO is from program GS-2018A-ToO-1 and is the $25^{th}$ ToO event in the simulation, it will have the observation identifier GS-2018A-ToO-1-25.  \\

The simulator distinguishes between user priorities of \textit{Interrupt Target of Opportunity} and all other target of opportunity types.  \textit{Standard Target of Opportunity} and \textit{Rapid Target of Opportunity} types are handled in the same way but have different time constraints.  When an \textit{Interrupt} type ToO arrives to the queue, the simulator will attempt to schedule it immediately.  If the conditions and observation constraints permit this, the plan changes to observe the ToO.  Otherwise, it does not get scheduled at all(due to a very short time constraint).  When \textit{Rapid} and \textit{Standard} type ToOs arrive, they are added to the queue without interrupting the current observation.  Once the current observation is completed, the remainder of the night is rescheduled.  

\subsubsection{Example simulation}
\label{sec:examplesim}
The following is an example of the simulator handling incoming ToOs and changing weather conditions throughout the night of 2018-05-01.

\begin{lstlisting}
	---------------------------------------------------------------------
	                  Gemini Adaptive Queue Planning Tool
	---------------------------------------------------------------------
	
	Dates: 2018-05-01 to 2018-05-01
	Number of nights: 1
	Daylight savings time: False
	
	Observatory: 
		Site:          gemini_south
		Height:        2750.0000 m
		Longitude:     -70.7367 deg
		Latitude:      -30.2407 deg

	Options:
	--------
	1.        Log file                                         loggylog.log                            
	2.        ToO file                                         tootypes.txt                            
	3.        ToO probability                                  0.3                                     
	4.        Max. number of ToOs per night                    4                                       
	5.        Conditions (iq,cc,wv)                            (70%, 50%, Any)                         
	6.        Conditions from distribution type                None                                    
	7.        Wind conditions (dir, vel)                       (330.0deg, 5.0m/s)                      
	8.        Generate random wind conditions                  False                                   
	9.        Probability of condition change                  0.3                                     
	10.       Max. number of condition changes per night       4                                       
	11.       Show plan plots                                  False                                   
	12.       Show airmass plot of each plan iteration         False                                   
	13.       Show airmass plots of plan building up           False                                   
	14.       Show sky conditions plots                        False                                   
	15.       Show wind condition plots                        False                                   

	dir       Show files in current directory                                                      

 Press enter to run or select an option: 


				-- Generating plan for night of 2018-05-01 --
	
	Sky conditions (iq, cc, wv):       (70%, 50%, Any)
	Wind conditions (dir., vel.):      330.0 deg, 5.0 m / s
	
	Solar midnight (UTC):              2018-05-02 04:40
	Solar midnight (local):            2018-05-02 00:40
	
	Sun ra:                            39.05 deg
	Sun dec:                           15.28 deg
	
	Moon ra:                           246.0 deg
	Moon dec:                          -16.78 deg
	Moon fraction:                     0.95
	Moon phase:                        0.44 rad
	
	
	                     -- 2018-05-01 schedule (iteration 1) --
	
	Obs. ID       Target      RA     Dec.   Instr  UTC    LST    Start  End    Dur.   AM    HA     Completed 
	-------       ------      --     ---    -----  ---    ---    -----  ---    ---    --    --     --------- 
	12 deg.twi.                                    23:00  8.95   19:00                                       
	A-LP-12-26    hen 3-225   133.9  -43.5  GPI    23:00  8.95   19:00  20:18  1.3 h  1.03  0.02   False     
	A-LP-12-81    wray 15-53  154.0  -57.9  GPI    00:18  10.25  20:18  21:42  1.4 h  1.13  -0.01  False     
	-FT-103-29    hip 64150   197.4  5.11   GPI    02:18  12.26  22:18  23:06  0.8 h  1.26  -0.90  False     
	A-Q-109-16    HIP65426    201.4  -51.6  GPI    03:06  13.06  23:06  00:30  1.4 h  1.08  -0.37  False     
	-FT-103-33    hip67620    208.1  -24.4  GPI    04:30  14.46  00:30  01:18  0.8 h  1.01  0.59   False     
	A-Q-133-25    WISE J1623  246.2  -5.18  GMOS-  05:18  15.27  01:18  02:24  1.1 h  1.15  -1.15  False     
	A-Q-103-23    HR Del      310.8  19.23  GMOS-  09:54  19.88  05:54  06:24  0.5 h  1.58  -0.84  False     
	12 deg. twi.                                   10:18  20.28  06:18                                       
	
	At 21:12 local time, Interrupt Target of Opportunity added to queue.
	
	
	                     -- 2018-05-01 schedule (iteration 2) --
	
	Obs. ID       Target      RA     Dec.   Instr  UTC    LST    Start  End    Dur.   AM    HA     Completed 
	-------       ------      --     ---    -----  ---    ---    -----  ---    ---    --    --     --------- 
	12 deg.twi.                                    23:00  8.95   19:00                                       
	A-LP-12-26    hen 3-225   133.9  -43.5  GPI    23:00  8.95   19:00  20:18  1.3 h  1.03  0.02   True      
	A-LP-12-81    wray 15-53  154.0  -57.9  GPI    00:18  10.25  20:18  21:18  1.0 h  1.13  -0.01  False     
	018A-T-1-1    SN-4        200.2  -40.1  GMOS-  01:18  11.26  21:18  23:18  2.0 h  1.13  -2.10  False     
	A-Q-133-25    WISE J1623  246.2  -5.18  GMOS-  03:18  13.26  23:18  00:24  1.1 h  1.59  -3.15  False     
	-FT-103-33    hip67620    208.1  -24.4  GPI    04:24  14.36  00:24  01:12  0.8 h  1.01  0.49   False     
	-FT-103-23    hip 67620   208.1  -24.4  GPI    05:12  15.17  01:12  02:00  0.8 h  1.05  1.29   False     
	A-Q-103-23    HR Del      310.8  19.23  GMOS-  09:54  19.88  05:54  06:24  0.5 h  1.58  -0.84  False     
	12 deg. twi.                                   10:18  20.28  06:18                                       
	
	At 23:36 local time, Standard Target of Opportunity added to queue.
	
	
	                     -- 2018-05-01 schedule (iteration 3) --
	
	Obs. ID       Target      RA     Dec.   Instr  UTC    LST    Start  End    Dur.   AM    HA     Completed 
	-------       ------      --     ---    -----  ---    ---    -----  ---    ---    --    --     --------- 
	12 deg.twi.                                    23:00  8.95   19:00                                       
	A-LP-12-26    hen 3-225   133.9  -43.5  GPI    23:00  8.95   19:00  20:18  1.3 h  1.03  0.02   True      
	A-LP-12-81    wray 15-53  154.0  -57.9  GPI    00:18  10.25  20:18  21:18  1.0 h  1.13  -0.01  False     
	018A-T-1-1    SN-4        200.2  -40.1  GMOS-  01:18  11.26  21:18  23:18  2.0 h  1.13  -2.10  True      
	A-Q-133-25    WISE J1623  246.2  -5.18  GMOS-  03:18  13.26  23:18  00:24  1.1 h  1.59  -3.15  True      
	018A-T-1-2    SN-6        180.2  -41.1  GMOS-  04:24  14.36  00:24  02:00  1.6 h  1.16  2.35   False     
	A-Q-103-23    HR Del      310.8  19.23  GMOS-  09:54  19.88  05:54  06:24  0.5 h  1.58  -0.84  False     
	12 deg. twi.                                   10:18  20.28  06:18                                       
	
	At 03:36 local time, sky conditions change to iq=70%, cc=50%, wv=20%.
	
	
	                                   -- 2018-05-01 schedule --
	
	Obs. ID       Target      RA     Dec.   Instr  UTC    LST    Start  End    Dur.   AM    HA     Completed 
	-------       ------      --     ---    -----  ---    ---    -----  ---    ---    --    --     --------- 
	12 deg.twi.                                    23:00  8.95   19:00                                       
	A-LP-12-26    hen 3-225   133.9  -43.5  GPI    23:00  8.95   19:00  20:18  1.3 h  1.03  0.02   True      
	A-LP-12-81    wray 15-53  154.0  -57.9  GPI    00:18  10.25  20:18  21:18  1.0 h  1.13  -0.01  False     
	018A-T-1-1    SN-4        200.2  -40.1  GMOS-  01:18  11.26  21:18  23:18  2.0 h  1.13  -2.10  True      
	A-Q-133-25    WISE J1623  246.2  -5.18  GMOS-  03:18  13.26  23:18  00:24  1.1 h  1.59  -3.15  True      
	018A-T-1-2    SN-6        180.2  -41.1  GMOS-  04:24  14.36  00:24  02:00  1.6 h  1.16  2.35   True      
	A-Q-103-23    HR Del      310.8  19.23  GMOS-  09:54  19.88  05:54  06:24  0.5 h  1.58  -0.84  True      
	12 deg. twi.                                   10:18  20.28  06:18                                       

Simulation complete!
\end{lstlisting}

The various iterations of the plan are provided, and plots of each plan may be viewed.  As seen above, an $Interrupt$ type ToO caused an immediate interruption of the plan at 12:12, whereas the $Standard$ type ToO was added to the queue at 23:36 and was scheduled later in the night at 00:24.  At 03:36, the viewing conditions improved so the plan was not interrupted.  Figure \ref{fig:planiters} shows plan iteration plots for this example.

{\centering
 \includegraphics[width=\columnwidth]{planiters}\\
 \captionof{figure}{Iterations of nightly plan with ToOs and changing conditions}\label{fig:planiters}
}
\vspace{4mm}

{\centering
 \includegraphics[width=\columnwidth]{skywind}\\
 \captionof{figure}{Sky conditions(left) and wind conditions(right) for the 2018-05-01 example}\label{fig:skywind}
}
\vspace{4mm}

\section{Weight Function Plotting Tool}
\label{sec:wfpt}

The Weight Function Plotting Tool functions as an independent program.  It requires only an observation file, program file, and instrument schedule file.  However, similar to the GQPT, the user must define the date and observatory in the command line if non-default parameters are desired. 

\subsection{Help guide}

\begin{lstlisting}
                                        Weight function plotting tool
*****************************************************************************************************               

    otfile                  OT catalog file name.

    prfile                  Gemini exechours program status file name.

    instcal                 Instrument calendar filename.

    -o   --observatory      Observatory site [DEFAULT='gemini_south']. Accepts the following:
                            1. 'gemini_north' (or 'MK' for Mauna Kea)
                            2. 'gemini_south' (or 'CP' for Cerro Pachon)

    -d   --date             Date 'YYYY-MM-DD' [DEFAULT=current].

    -dst --daylightsavings  Toggle daylight savings time [DEFAULT=False].

    -dt  --gridsize         Size of time-grid spacing [DEFAULT=0.1hr].

                            Sky conditions:
    -i   --iq               Image quality constraint [DEFAULT=70].
    -c   --cc               Cloud cover constraint   [DEFAULT=50].
    -w   --wv               Water vapor constraint   [DEFAULT=Any].

                            Wind conditions:
    -dir --direction        Wind direction [DEFAULT=270deg].
    -vel --velocity         Wind velocity [DEFAULT=10deg].

    -rw  --randwind         Random wind conditions (use mean and standard deviation of site):
                                Cerro Pachon : dir=(330 +/- 20)deg, vel=(5 +/- 3)m/s
                                Mauna Kea     : dir=(330 +/- 20)deg, vel=(5 +/- 3)m/s

    -u   --update           Download up-to-date IERS(International Earth Rotation and Reference                            Systems).

*****************************************************************************************************                        

positional arguments:
  otfile
  prfile
  instfile

optional arguments:
  -h, --help            show this help message and exit
  -o OBSERVATORY, --observatory OBSERVATORY
  -d DATE, --date DATE
  -dst, --daylightsavings
  -dt GRIDSIZE, --gridsize GRIDSIZE
  -iq IQ, --iq IQ
  -cc CC, --cc CC
  -wv WV, --wv WV
  -dir DIRECTION, --direction DIRECTION
  -vel VELOCITY, --velocity VELOCITY
  -rw, --randwind
  -u, --update

\end{lstlisting}

\subsection{Plotting}
\label{sec:wfptplotting}
The WFPT, just like the GQPT, defaults to the current night if not specified otherwise.  Here is an example of how it would be used...
\begin{lstlisting}
>>> python wfpt.py observations.txt execHours.txt instschedule.txt
\end{lstlisting}

The program will generate the necessary data structures before displaying the options menu and prompting the user.
\begin{lstlisting}
   ---------- Weight function plotting mode ----------

	Plan date:          2018-04-01

	Options:
	--------
	1.   See list of available observations      -                                       
	2.   Conditions (iq,cc,wv)                   (0.7, 0.5, 1.0)                         
	3.   Wind conditions (dir, vel)              (330 deg, 5 m / s)       

	Select option or provide an observation identifier:
\end{lstlisting}

If option '1' is selected, the list of available observations is displayed.
\begin{lstlisting}
	Observation            Program              Target               Group                                   
	-----------            -------              ------               -----                                   
	GS-2018A-A-12-2        GS-2018A-A-12        OGLE-TR-7            OGLE-TR-7 - [1] Visitor IGRINS          
	GS-2018A-A-10-13       GS-2018A-A-10        HIP65426             HIP65426 - [3] GPI Prism Coronograph
	GS-2018A-A-10-16       GS-2018A-A-10        HIP65426             HIP65426 - [1] GPI Prism Coronograph
	GS-2018A-B-10-18       GS-2018A-B-10        HIP65426             HIP65426 - [4] GPI Prism Coronograph
	GS-2018A-B-10-20       GS-2018A-B-10        HIP65426             HIP65426 - [5] GPI Prism Coronograph
	GS-2018A-B-10-23       GS-2018A-B-10        HIP65426             HIP65426 - [2] GPI Prism Coronograph
	GS-2018A-B-13-37       GS-2018A-B-13        Alpha Centauri       Alpha Centauri A - [visit 1] GPI
	GS-2018A-C-15-41       GS-2018A-C-15        Alpha Centauri       Alpha Centauri A - [visit 2] GPI
	GS-2018A-C-15-47       GS-2018A-C-15        alpha centauri       Alpha Centauri B - [visit 3] GPI
	GS-2018A-C-15-49       GS-2018A-C-15        alpha centauri       Alpha Centauri B - [visit 4] GPI
	...                    ...                  ...                  ...
\end{lstlisting}

The user may then select an observation by providing the full observation identifier. The software will display the observation weighting function, variables.  Constants are printed to the terminal and time dependent variables are plotted (Figures \ref{fig:weightcomp1}, \ref{fig:weightplot}).  Additionally, the sky brightness condition can be examined for the provided viewing conditions (Figure \ref{fig:skyplot2}).

\begin{lstlisting}
>>> Select option or provide an observation identifier: GS-2018A-Q-109-13

	GS-2018A-Q-109-13 weights
	-------------------------
	Total cond:              12.915451895 (iq=0.7, cc=0.5, bg=1.0, wv=1.0)
	RA:                      7.757871808036047
	Band:                    3000.0 (Band 1)
	User priority:           1.0 (Medium priority)
	Status:                  1.0 (Partially complete: prog=False, obs=False)
\end{lstlisting}

{\centering
 \includegraphics[width=0.8 \columnwidth]{weight_comps_1}\\
 \captionof{figure}{Time dependent weighting factors for GS-2018A-Q-109-13 on 2018-04-01}\label{fig:weightcomp1}
}
\vspace{4mm}

{\centering
 \includegraphics[width=0.8 \columnwidth]{weight_func_plot}\\
 \captionof{figure}{GS-2018A-Q-109-13 weighting function}\label{fig:weightplot}
}
\vspace{4mm}

{\centering
 \includegraphics[width=0.8 \columnwidth]{skycond_plot_2}\\
 \captionof{figure}{GS-2018A-Q-109-13 sky conditions (including sky background)}\label{fig:skyplot2}
}
\vspace{4mm}

{\centering
 \includegraphics[width=0.8 \columnwidth]{windcond_plot}\\
 \captionof{figure}{GS-2018A-Q-109-13 wind conditions}\label{fig:windplot2}
}
\vspace{4mm}

\section{Input files}
\label{sec:fileformats}
The following file formats are currently required by the GQPT and WFPT.  These formats may change in the future as new methods become available for querying observation, program, ToO, and instrument information.

\subsection{Observation file}
The observation catalog file uses the same format as the ODB browser ascii output.  

\subsection{Target of opportunity file}
This file uses the same format as the observation file.  However, the start times of the observation time constraints must be assigned the value -1.  This is necessary in order for the software to start the ToO time constraint at the time that is it generated.   Additionally, the observation identifiers are set to 'null'.  ToOs are assigned an observation number when they are generated. 

\subsection{Program file}
The GQPT and WFPT require any version of an \textit{execHour} program status file.  Here is an example of the file format...
\begin{lstlisting}
# Executed Hours Report for Gemini South 2018A
# Generated at 01 Apr 2018 00:00:00 GMT.
# Columns: Program ID,Allocated,Elapsed,Non-charged,Partner,Program
GS-2018A-A-1,20.00,0.00,0.00,0.00,0.00
GS-2018A-B-11,8.00,0.00,0.00,0.00,0.00
GS-2018A-B-12,2.00,0.00,0.00,0.00,0.00
GS-2018A-C-1,8.70,0.00,0.00,0.00,0.00
GS-2018A-C-19,10.00,0.00,0.00,0.00,0.00
GS-2018A-D-25,11.20,0.00,0.00,0.00,0.00
GS-2018A-D-26,0.30,0.00,0.00,0.00,0.00
...
\end{lstlisting}


\subsection{Instrument schedule}
Currently, the instrument schedule must be manually written.  This file contains the installed instruments, as well as the GMOS disperser, GMOS focal plane unit, and Flamingos2 focal plane unit installed on each night.  If instrument information is missing for any dates in the scheduling period, the software will allow any instrument and configuration to be scheduled. The following is an example of the instrument file format - note that columns must be separated by tab '\textbackslash t'  characters.
\begin{lstlisting}
UT	Instrument	GMOS FPU	GMOS Disperser	F2 FPU
----	-----	--------	--------------	------
2018-04-01	GPI,cal,GMOS-S,INGRINS,Flamingos2,Visitor Instrument	null	null	null
2018-04-02	GPI,cal,GMOS-S,INGRINS,Flamingos2,Visitor Instrument	null	null	null
2018-04-03	GPI,cal,GMOS-S,INGRINS,Flamingos2,Visitor Instrument	null	null	null
2018-04-04	GPI,cal,GMOS-S,INGRINS,Flamingos2,Visitor Instrument	null	null	null
2018-04-05	GPI,cal,GMOS-S,INGRINS,Flamingos2,Visitor Instrument	null	null	null
...
\end{lstlisting}

\section{Data structures}

The GQPT and WFPT use $astropy.table.Table$ objects as the main data structure.  Tables or list of tables store important times, time grids, Sun data, Moon data, observation information, instrument information, program information, and target data for the entire scheduling period.

\begin{table}[h!]
\caption{Main data structures used in GQPT, WFPT}\label{tab:vars}
\centering
\centering
\resizebox{\columnwidth}{!}{%
\begin{tabular}{p{2cm}p{4cm}p{8cm}}
Variable  & Data type & Description \\\hline
obs & astropy.table.Table & All observations catalog file\\
progs & astropy.table.Table & All Gemini programs in execHours file\\
timetable & astropy.table.Table & Important times and time-grids for scheduling period\\
sun & astropy.table.Table & Location of Sun at solar midnights throughout scheduling period\\
moon & astropy.table.Table & Location of Moon at at all times throughout scheduling period\\
targetcal & list of astropy.table.Table & Locations and important quantities for  throughout nights in scheduling period\\
\end{tabular}
}
\end{table}

\subsection{Observation table}

A single observation table stores the entire queue (Table \ref{tab:obs}).  When ToOs are generated they are appended to the bottom of this table. 

\subsection{Time, Sun, and Moon tables}

The time, instrument, Sun, and Moon variables are each a single astropy table (Tables \ref{tab:timetable}, \ref{tab:sun}, \ref{tab:moon}, \ref{tab:insts}).  The rows in these tables corresponding to single nights in the scheduling period. 

\subsection{Target tables}

Target data is stored as a list of astropy tables.  Each table in the list corresponds to a night in the scheduling period.  The rows in each of these tables corresponds to the available targets for that night.  Table \ref{tab:targets} shows the structure of each of these tables.  It should be noted that not all columns constructed simultaneously.  The columns i, id, ZD, HA, AZ, AM, and mdist are computed when the table is first constructed.  Once the scheduling simulation begins, the software will add the columns vsb, bg, and weight, as these columns are dependent on the current conditions and status of the queue. 

\begin{table}[h!]
\caption{Observation table (variable name: $obs$)}\label{tab:obs}
\centering
\centering
\resizebox{\columnwidth}{!}{%
\begin{tabular}{p{2cm}p{3cm}p{9cm}}
Key  & Data type & Description \\\hline
prog\_ref & str & unique program reference\\
obs\_id & str & unique observation identifier\\
pi & str & principle investigator\\
inst & str & instrument\\
target & str & source name\\
ra & astropy.coordinates & right ascension\\
dec & astropy.coordinates & declination\\
band & int & ranking band (1,2,3,4)\\
partner & str & Gemini partner name\\
obs\_status & str & 'ready' status\\
obs\_time & astropy.units hours & hours observed\\
charged_time & astropy.units hours & hours charged\\
obs\_comp & float & completion fraction\\
obs\_class & str & observation class\\
iq & float & image quality percentile\\
cc & float & cloud condition percentile\\
bg & float & sky background percentile\\
wv & float & water vapour percentile\\
user\_prior & str & user priority \newline ('Low', 'Med', 'High', 'Interrupt Target of Opportunity', 'Rapid Target of Opportunity', 'Standard Target of Opportunity')\\
group & str & observation group name\\
elev\_const & dict & elevation constraint \newline ('{type min max}')\\
time\_const & dict & time constraint \newline ('[\{start duration repeats period\}]')\\
ready & bool & 'ready' status\\
disperser & str & disperser\\
fpu & str & focal plane unit\\
grcwlen & str & grating control wavelength\\
crwlen & str & central wavelength\\
filter & str & filter\\
mask & str & mask\\
xbin & str & bin number\\
ybin & str & bun number\\
\end{tabular}
}
\end{table}

\begin{table}[h!]
\caption{Time table (variable name: $timetable$)}\label{tab:timetable}
\centering
\centering
\resizebox{\columnwidth}{!}{%
\begin{tabular}{p{3cm}p{5cm}p{6cm}}
Key  & Data type & Description \\\hline
date & string & date of night in schedule\\
utc & astropy.time.core.Time array & UTC time grid\\
local & astropy.time.core.Time array & local time grid \\
lst & astropy.units hourangle array & local sidereal time grid\\
evening\_twilight & astropy.time.core.Time & nautical twilight time\\
morning\_twilight & astropy.time.core.Time & nautical twilight time\\
solar_midnight & astropy.time.core.Time & solar midnight time\\
\end{tabular}
}
\end{table}

\begin{table}[h!]
\caption{Sun table (variable name: $sun$)}\label{tab:sun}
\centering
\centering
\resizebox{\columnwidth}{!}{%
\begin{tabular}{p{2cm}p{5cm}p{7cm}}
Key  & Data type & Description \\\hline
ra & astropy.units degrees & right ascension at solar midnight\\
dec & astropy.units degrees & declination at solar midnight\\
ZD & astropy.units degrees & zenith distance angles\\
HA & astropy.units hourangle & hour angles\\
AZ & astropy.units radians & azimuth angles\\
\end{tabular}
}
\end{table}

\begin{table}[h!]
\caption{Moon table (variable name: $moon$)}\label{tab:moon}
\centering
\centering
\resizebox{\columnwidth}{!}{%
\begin{tabular}{p{2cm}p{5cm}p{7cm}}
Key  & Data type & Description \\\hline
fraction & float & fraction illuminated at solar midnight\\
phase & astropy.units radians & phase angle at solar midnight\\
ra\_mid & astropy.units degrees & right ascension at solar midnight\\
dec\_mid & astropy.units degrees & declination at solar midnight\\
ra & astropy.units degrees array & right ascension throughout night\\
dec & astropy.units degrees array & declination throughout night\\
ZD & astropy.units degrees array & zenith distance angles\\
HA & astropy.units hourangle array & hour angles\\
AZ & astropy.units radians array & azimuth angles\\
AM & float array & airmass throughout night\\
\end{tabular}
}
\end{table}

\begin{table}[h!]
\caption{Instrument calendar table (variable name: $instcal$)}\label{tab:insts}
\centering
\centering
\resizebox{\columnwidth}{!}{%
\begin{tabular}{p{2cm}p{5cm}p{7cm}}
Key  & Data type & Description \\\hline
date & string & date of night in schedule\\
insts & string & list of available instruments\\
gmos_fpu & string & available GMOS focal plane unit\\
gmos_disp & string & available GMOS disperser\\
f2_fpu & string & available Flamingos-2 focal plane unit \\
\end{tabular}
}
\end{table}

\begin{table}[h!]
\caption{Target table (variable name: $targets$) - each of these tables is an element from a list of tables: $targetcal$.}\label{tab:targets}
\centering
\centering
\resizebox{\columnwidth}{!}{%
\begin{tabular}{p{2cm}p{5cm}p{7cm}}
Attribute  & Data type & Description \\\hline
i & integer & observation table row index\\
id & string & observation identifier\\
ZD & astropy.units degrees array & zenith distance angles\\
HA & astropy.units hourangle array & hour angles\\
AZ & astropy.units radians array & azimuth angles\\
AM & float array & airmass throughout night\\
mdist & astropy.units radians & angular distances from moon\\
vsb & float array & visible sky brightnesses\\
bg & float array & sky background condition percentiles\\
weight & float array & weighting factors\\
\end{tabular}
}
\end{table}

\end{document}



